  \section{Introduction}

  \subsection{What is Metadata?}

  While encryption technology like PGP or S/MIME provides a secure way to protect content from prying eyes, ever since Edward Snowdens whistleblowing we learned that metadata --- most notably information about who communicates with whom --- is equally interesting and much easier to analyze.

  There are a few examples where meta data might be enough to get you in trouble. If you write to someoune in the IS, you might not be able to fly the next time you want to visit the U.S. The no-fly list doesn't care if you're a journalist, or had no clue that this person was a terrorist.

  If Samsung knows Apple talks excessively with the sole producer of this nifty little sensor, they don't need the details --- the S7 will sport one of those, too. (Failing to see that Apple used it to build a car.)

  \subsection{How Can We Hide Metadata?}

  With e-mail, we can only prevent this by encrypting the connection to the server as well as between servers. Therefore we can only hope that both our and the recipient's e-mail provider are both trustworthy as well as competent.\footnote{Of course they should be free as well.}

  With Bitmessage we send a message to a sufficiently large number of participants, with the intended recipient among them. Content is encrypted such as only the person in possesion of the private key can decrypt it. All participants try to do this in order to find their messages.